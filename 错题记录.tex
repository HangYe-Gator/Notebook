%%%%%%%%%%%%%%%%%%%%%%%%%%%%%%%%%%%%%%%%%%%%%%%%%%%%%%%%%%%%%%%
%
% Welcome to Overleaf --- just edit your LaTeX on the left,
% and we'll compile it for you on the right. If you open the
% 'Share' menu, you can invite other users to edit at the same
% time. See www.overleaf.com/learn for more info. Enjoy!
%
%%%%%%%%%%%%%%%%%%%%%%%%%%%%%%%%%%%%%%%%%%%%%%%%%%%%%%%%%%%%%%%
\documentclass{ctexart}
\usepackage[svgnames]{xcolor}
\usepackage{amsmath}
\newcommand{\cbox}[2][yellow]{%
  \colorbox{#1}{\parbox{\dimexpr\linewidth-2\fboxsep}{\strut #2\strut}}%
}



\begin{document}
\pagestyle{empty}
\section{AMS510期末考试题}
\begin{enumerate}

    \item 
        \[ \int_{-\infty}^{+\infty} e^{-2x^2+8x} \,dx \]
    \cbox[gray!30]{\textsl{配方失误}}

    配方的过程:\[-2x^{2}+8x\]第1步:提出$a$项的系数,为-2,于是原式变成了\[-2(x^{2}-4x)\]第2步:把$b$项的系数除以2再平方,然后把得到的数分为正数和负数放在后面,即:
    \newline 把$b$项的系数除以2再平方:$-4/2=-2,-2^2=4$
    \newline 然后把得到的数分为正数和负数放在后面,即为:\[-2(x^{2}-4x+4-4)\]
    
    第3步:把负的那一项系数乘以a的系数后,放在括号外面,即$-4*-2=8$,把8放在外面,变成:\[-2(x-4x+4)+8\]第4步:完成配方:\[-2(x-2)^2+8\]
    
    \item 
     手推Taylor Expansion:\[y=\frac{1}{(1-x)^2}\]
     \cbox[gray!30]{\textsl{老师自己改错了,但是我也侥幸套公式对了}}
     问题:Solution 上,既然$\frac{1}{(1-x)}'=\frac{1}{(1-x)^2}$,并且我们也已经知道\[ \frac{1}{(1-x)}=\sum_{n=0}^{\infty} x^n\], 那么为什么我们不能直接对$\sum_{n=0}^{\infty} x^n$ 进行求导得出:\[ \frac{1}{(1-x)^2}= \sum_{n=0}^{\infty} nx^{n-1}\]

    \item 
    Find the Maximum of the function:\newline
    $f(x,y,z)=xyz$\qquad subject to \qquad$4x+2y+z=6$ \newline
    \cbox[gray!30]{\textsl{计算失误,无法解出联立方程组,我需要去寻找拉格朗日的复杂题、怪题。}}

    \item
    calculate the limits \[  \lim\limits_{(x,y)\rightarrow (0,0)}{\frac{sin(xy^{2})}{x^{2}y}} \]
    \cbox[gray!30]{计算失误} 
    \[  \lim\limits_{(x\rightarrow 1)}\frac{1-4sin^2{\frac{\pi x}{6}}}{1-x^2} \]
    \cbox[gray!30]{求导失误} 

    \item
    三重积分题目,计算体积:
    \[ V:{(x,y,z)\in R^3, x^2+y^2+4z^2 \leq 1} \]
    1.注意判断,此时此刻用于区域的小于等于1,所以我们的R是在0~1区间内变化的,如果被积的区域是等于1,那么我们的R肯定是固定的,所以这里的r要注意,是动态变化的而不是固定的(会影响后面代入公式,如果认为r是固定的,则为代入1,那就错了)\newline
    2. 根据体积的计算公式,进行极坐标变化
    \newline 首先,三重积分中,如果要让你计算体积,那么f(x,y,z)=1,所以三重积分的体积几何意义表示为:
    \[\iiint_V f(x,y,z) \,dx\,dy\,dz = \iiint_V \,dx\,dy\,dz  \] 
    其中根据椭圆坐标变换分别代入$x$,$y$,$z$,极坐标公式变换为:
    \[
    \left\{
    \begin{aligned}
    x=arsin \theta cos\phi \\
    y=brsin \theta cos\phi \\
    z=crsin \theta cos\phi \\
    J=r^2sin \theta
    \end{aligned}
    \right.
    \]
其中根据椭球体的公式可知:a=1,b=1,c=$\frac{1}{4}$\newline
   
\cbox[gray!30]{这道题完善一下过程}

    \item 
    计算矩阵
    \[ A=
    \begin{bmatrix}
    3 & 2 & 4 \\
    2 & 0 & 2 \\
    4 & 2 & 3
    \end{bmatrix}
    \] 的eigenvalue and eigenvector,Diagonalization of A.($P^{-1}AP$)
    这道题判断失误了,并且要记住,满秩矩阵有多少rank就应该要有多少的$\lambda$.这道题计算失误。
    \cbox[gray!30]{\textsl{To-Do:规范一下对角化的过程}}
    
    \item 
    证明不等式:\[
    (\int_{a}^{b} |f(x)|dx)(\int_{a}^{b}\frac{1}{|f(x)|}dx) \leq (b-a)^2
    \]
    Cauchy–Schwarz inequality 一定要记住:
    
    
\end{enumerate}

\section{第二次课程}
\begin{enumerate}
    \item 二重极限的计算
    \[  \lim\limits_{(x,y)\rightarrow (0,0)}{\frac{1+x^{2}+y^{2}}{x^{2}+y^{2}}} \]
    \cbox[gray!30]{\textsl{不懂变形,苦苦计算y=mx的极限}}
    首先,把原式拆成
    \[1+\frac{1}{x^2+y^2}\]
    然后我们再去观察,当$(x,y) \rightarrow(0,0)$ 时,$\frac{1}{x^2+y^2} \rightarrow \frac{1}{0} \rightarrow \infty$,故$1+\infty=\infty$. 
    所以极限值就为$\infty$

    \item  \[  \lim\limits_{(x,y)\rightarrow (0,0)}{(x+y) \cdot \sin{\frac{1}{x^{2}+y^{2}}}} \]
     \cbox[gray!30]{\textsl{不知道使用放缩(虽然现在一看到三角函数的极限就想到放缩)}}

    首先$\sin{\frac{1}{x^2+y^2}}\leq 1$,我们把整个式子结合起来就可以建立一个“三明治”:
    \begin{equation}
    \tag{1}
    0\leq |(x+y)\sin{\frac{1}{x^2+y^2}} \leq |x+y|
    \end{equation}
    又因为$(x+y) \rightarrow (0,0)$,所以我们可以另外得到:
    \[
    0\leq |(x+y)\sin{\frac{1}{x^2+y^2}} \leq |x+y| \rightarrow 0
    \]
    所以极限为0. \\ 
    为什么(1)中的公式里,右边是:$\leq |x+y|$,因为:sin 函数是一个有界量,它一定是小于等于1的,所以我们直接让sin函数取得最大值,接下来的唯一能够变的只剩下x+y了,所以右边就是$\leq |x+y|$,当然了,如果你可以一步到位,直接观察x+y为0,0乘以任何数都为0似乎也可以?
    
\end{enumerate}



\end{document}
